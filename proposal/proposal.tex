\documentclass[11pt]{article}
\bibliographystyle{siam}

\title{Classification of fMRI Data}
\author{
  Escobar, Giancarlo\\
  \texttt{giancarloescobar}
  \and
  Pimentel, Noel\\
  \texttt{noelpimentel}
  \and
  Yousuf, Imran\\
  \texttt{imranyousuf}
  \and
  Zheng, Chang\\
  \texttt{changzheng1993}
}

\begin{document}
\maketitle

The functional architecture of the object vision pathway in the human brain was investigated using fMRI imaging to measure patterns of response in ventral temporal cortex while subjects viewed categories of objects and nonsense pictures in Haxby et al. [Science 293 (2001) 2425]  . Haxby argued that category related responses in the VT lobe during visual object identification were overlapping and distributed in topography. At the time of Hanson et al. [Elsevier 23 (2004) 156] \cite {hanson2004combinatorial} there were prevailing views that objects codes were focal and localized to specific areas like the fusiform and the parahippocampal gyri.  In this paper, Hanson et al. provide a crucial test of Haxby's hypothesis, on our data \cite {haxby2001vor}, using a neural network classifier that detects more general topographic representations and achieves an 83 percent correct generalization performance on patterns of voxel responses in out-of-sample tests (i.e. testing data).  

Our approach is intended to surpass this performance.  We will use similar methods used by Hanson et al. (i.e. neural networks and voxel-wise sensitivity analysis) while implementing different techniques along the way (i.e. bootstrap, cross validation, stacking). We also intend on implementing some different classification methods like LDA, QDA, trees (bagging, random forests, boosting), SVMs, etc. in order to compare method performances and gain further insight on the fMRI data.

\bibliography{proposal}

\end{document}